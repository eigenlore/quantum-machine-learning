\documentclass{beamer}
\usepackage[utf8]{inputenc}
\usetheme{Singapore}
\usecolortheme{default}
\beamertemplatenavigationsymbolsempty


\title[Genetic algorithm for QSVM] 
{Genetic algorithm for Quantum Support Vector Machines}



\author[Lorenzo Tasca]
{Lorenzo Tasca}
 

\date[25/11/2024] 
{25 Novembre 2024}

\logo{\includegraphics[height=1cm]{logo.png}}


\begin{document}

\frame{\titlepage}




\begin{frame}
\frametitle{Introduzione}
    \begin{itemize}
    \item<1-> Quantum Machine Learning si propone di sfruttare le potenzialitò del
    Quantum Computing per potenziare le performance di alcuni algoritmi di Machine Learning. 
        \item<2-> Vedremo come il potenziale di Coulomb fa sorgere vari problemi a causa della sua natura a lungo raggio, e necessita quindi di una trattazione a parte.
        \end{itemize}
\end{frame}

\begin{frame}
\frametitle{Sezione d'urto differenziale}



   
 Per definirla partiamo dall'equazione di Lippman-Scwhinger, che ci dice come un'onda piana nell'urto viene deformata in un'onda sferica:
\begin{columns}
\column{0.5\textwidth}
  $$\psi\approx e^{ik\cdot x}+f(k,k')\frac{e^{ikr}}{r}.$$


\column{0.5\textwidth}
\begin{figure}
      \includegraphics[height=2.5cm]{logo.png}
 \end{figure}


\end{columns}
\vspace{4mm}
    Richiamando il caso classico, possiamo definire la sezione d'urto $d\sigma$ come la sezione del fascio incidente attraverso la quale passa una corrente di probabilità pari a quella che entra in un angolo solido $d\Omega$:  $$J_{in}\,\alert{d\sigma}=J_{out}\,d\Sigma,$$ con $d\Sigma=r^2d\Omega$.
    Si trova che  $$\frac{{d\sigma}}{d\Omega}=|f(k,k')|^2.$$

\end{frame}







\end{document}